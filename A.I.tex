%%%%%%%%%%%%%%%%%%%%%%%%%%%%%%%%%%%%%%%%%%%%%%%%%%%%%%%%%%%%%%%%%%%%%%%%%%%%%%%%
%2345678901234567890123456789012345678901234567890123456789012345678901234567890
%        1         2         3         4         5         6         7         8

\documentclass[letterpaper, 10 pt, conference]{ieeeconf}  % Comment this line out
                                                          % if you need a4paper
%\documentclass[a4paper, 10pt, conference]{ieeeconf}      % Use this line for a4
                                                          % paper

\IEEEoverridecommandlockouts                              % This command is only
                                                          % needed if you want to
                                                          % use the \thanks command
\overrideIEEEmargins
% See the \addtolength command later in the file to balance the column lengths
% on the last page of the document



% The following packages can be found on http:\\www.ctan.org
%\usepackage{graphics} % for pdf, bitmapped graphics files
%\usepackage{epsfig} % for postscript graphics files
%\usepackage{mathptmx} % assumes new font selection scheme installed
%\usepackage{times} % assumes new font selection scheme installed
%\usepackage{amsmath} % assumes amsmath package installed
%\usepackage{amssymb}  % assumes amsmath package installed
\usepackage[absolute]{textpos}
\usepackage{comment}

\title{\line(1,0){350}\\How will Artificial Intelligence change the world?\\\line(1,0){350}}

%\author{ \parbox{3 in}{\centering Huibert Kwakernaak*
%         \thanks{*Use the $\backslash$thanks command to put information here}\\
%         Faculty of Electrical Engineering, Mathematics and Computer Science\\
%         University of Twente\\
%         7500 AE Enschede, The Netherlands\\
%         {\tt\small h.kwakernaak@autsubmit.com}}
%         \hspace*{ 0.5 in}
%         \parbox{3 in}{ \centering Pradeep Misra**
%         \thanks{**The footnote marks may be inserted manually}\\
%        Department of Electrical Engineering \\
%         Wright State University\\
%         Dayton, OH 45435, USA\\
%         {\tt\small pmisra@cs.wright.edu}}
%}

\author{Gary Mannion$^{}$, Student, GMIT $^{}$% <-this % stops a space
%\thanks{*This work was not supported by any organization}% <-this % stops a space
%\thanks{$^{1}$H. Kwakernaak is with Faculty of Electrical Engineering, Mathematics and Computer Science,
 %       University of Twente, 7500 AE Enschede, The Netherlands
  %      {\tt\small h.kwakernaak at papercept.net}}%
%\thanks{$^{2}$P. Misra is with the Department of Electrical Engineering, Wright State University,
 %       Dayton, OH 45435, USA
  %      {\tt\small p.misra at ieee.org}}%
}


\begin{document}

\begin{textblock}{9}(1,0.5)
\noindent\small Research Methods in Computing and IT, 4th Year Software Development
\end{textblock}


\maketitle
\thispagestyle{empty}
\pagestyle{empty}


%%%%%%%%%%%%%%%%%%%%%%%%%%%%%%%%%%%%%%%%%%%%%%%%%%%%%%%%%%%%%%%%%%%%%%%%%%%%%%%%
\begin{abstract} \newline
In this paper I will give an in-depth analysis of A.I (Artificial Intelligence). I will go in to detail the many ways that this new, emerging technology can have serious repercussions on our world as we know it today, but also the ways that it can have a very positive affect on our world in the future. A.I is quickly becoming one of the most talked about and popular technology out there with many big tech companies investing heavily in it as they see a future where A.I will be hugely involved. We live in a world where A.I is surrounding us even in places where we aren't even aware. \vspace{2mm}

\textbf{\textit{Keywords---}} Machine Learning, Neural Network, Robotics
\end{abstract}


%%%%%%%%%%%%%%%%%%%%%%%%%%%%%%%%%%%%%%%%%%%%%%%%%%%%%%%%%%%%%%%%%%%%%%%%%%%%%%%%
\section{INTRODUCTION}

A.I is the study and creation of machines that would require human intelligence in order to be operational. The main aim of A.I is to produce machines that are capable of making decisions that a human is also capable of. These machines should be able to solve problems, make decisions, hold the same intelligence as a human and learn on its own in initiative. 

Even though A.I has been around for centuries and has been seen in mythological discoveries it has only become popular in the last decade due to increased research and development. Everyones interpretation of A.I is that of a Terminator like robot that will one day become smarter than humans and will be able to control the world and humans could seize to exist. Although this "could" be a possible outcome it most likely would never happen. 

A.I is a very broad area of computer science which is commonly broken into 3 main types: Machine Learning, Neural Networks and Deep Learning. A.I uses algorithms in order to process an operation, an algorithm is a set of unambiguous instructions that a computer is able to execute or to perform an operation.

\section{History of A.I}
%\subsection{Selecting a Template (Heading 2)}
%\subsection{Maintaining the Integrity of the Specifications}
The idea of A.I hasn't just been a new technology thought about in the last few years or decades, this has been an idea that has been around for centuries dating all the way back to the ancient Greeks where there has been prove that they had myths about robots, also Chinese and Egyptian engineers developed automatons. The field of A.I itself wasn't initially founded until 1956 at a conference in Dartmouth College in New Hampshire where the term 'Artificial Intelligence was first mentioned. Many scientists who attended the conference were optimistic about the future of A.I including MIT cognitive scientist Marvin Minsky who is quoted to have said [1]“Within a generation the problem of creating ‘Artificial Intelligence’ will substantially be solved,” from the book “AI: The Tumultuous Search for Artificial Intelligence” (Basic Books, 1994). Moving forward from this foundation of 1956, achieving artificial intelligence was not as easy as first thought. After many reports of poor progress in development in A.I, all government funding and any popular interest in the field was dropped from 1974-1980, also known as the “A.I winter”. Later on in the 1980’s the field was refunded by the British government in order to compete with the Japanese efforts to research A.I. Once again a second “A.I winter” occurred from 1987 to 1993. This fall occurred as a reduced amount of government funding and also coinciding with collapse of the early general-purpose computers. 
It wasn’t until 1997 when research picked up again when IBM’s Deep Blue became the first computer to win a game of chess by beating chess champion Russian grandmaster Garry Kasparov. In the 1950s a computer scientist and Mathematician by the name of Alan Turing developed a competition to assess whether a machine is intelligent enough as a human. In 2018 the talking computer “chatbot” was able to trick judges into believing that it was a real skin-and-blood human during this Turing test. However this achievement has been dismissed by many A.I experts as they believe that only a third of the judges were fooled by the ‘chatbot’ and always claiming that it also was able to dodge many questions that a human would not be able to. 
In 2011, on an exhibition game of jeopardy, a quiz show, IBM’s question answering system, Watson, managed to defeat the two greatest Jeopardy champions, Brad Rutler and Ken Jennings, by a huge margin. Adding to this, in 2016, AlphaGo won four out of five games of Go, [2]an abstract strategy board game with the aim to surround more territory than your opponent. AlphaGo managed to beat Go champion Lee Sedol becoming the first computer to play the game to beat a professional Go player. In 2017, world number one Go player for two years running Ke Jie lost a three-game match against AlphaGo. These winnings saw the completion of a significant milestone in A.I development as Go is a very complex game, much more complex than chess. 


\section{Types of A.I}

A.I is a very broad concept to deal with it so it can be divided up into different types in order to gain a better understanding of the area of A.I. [3]Although there are various types of A.I to discuss I have decided to talk about the three most important types, Machine Learning, Neural Networks and Deep Learning. Other types of A.I would include Weak AI, Strong AI and Artificial Super-intelligence which simply categorize the different strength levels of A.I used, the more power required for the operation would be decided by choosing one of these types. 

\subsection{Machine Learning} 
Machine Learning can be described in the same way as A.I but the main difference between the two is that A.I is the broad science of copying human abilities, whereas machine learning is a subset of A.I that is able to teach a machine to be able to learn. [4] Machine learning is born from the theory that machines could learn without having to be manually programmed to perform a task and a pattern recognition system. In recent years machine learning has moved on to making a computer learn from data it can identify and when being exposed to this new data being able to adapt to it independently. A few examples of the huge developments in machine learning would the self-driving Google car, online pop-up recommendations for apps or stores and being able to see what is being said online about a certain topic or about a certain store you are looking at. 
The biggest change in making machine learning become so popular in recent times is the availability of affordable data storage to process the more powerful computers required to process the data. These improvements have made it so much easier to produce large scale models at a much faster, accurate and making complex data much simpler to deal with than the past. In order to create a good machine learning model you must have a proper algorithm written, make it scalable and it needs an automation process. Machine learning is becoming vastly used across all industries, some examples include: 
\begin{itemize}
\item Financial Services - Here a bank would use machine learning in order to process the large chunk of data they would be dealing with on a daily basis and also be able to keep track of any fraudulent activity as a security platform.
\end{itemize}

\begin{itemize}
\item Government - Countries would use this technology in order to identify ways in which they would be able save money and also it would be an efficient way to analyze all the data spread out between all the areas of the state.
\end{itemize}

\begin{itemize}
\item Transportation - This area would use machine learning in order to track patterns such as when is transport most popular during the day and when is it not so they know where they can save money by not having as much transport on the road at certain times during the day.
\end{itemize}

\begin{itemize}
\item Health care - New devices are being made readily available that have the capability  to track a persons heart rate, track steps and sleeping trends. Even these small devices are being used at a bigger scale in our hospitals making work easier on doctors and also improving our own health care system.
\end{itemize}



\subsection{Neural Network}

\subsection{Deep Learning}



\section{USING THE TEMPLATE}

\subsection{Headings, etc}

%\subsection{Figures and Tables}
 
\section{CONCLUSION}



\addtolength{\textheight}{-12cm}   % This command serves to balance the column lengths
                                  % on the last page of the document manually. It shortens
                                  % the textheight of the last page by a suitable amount.
                                  % This command does not take effect until the next page
                                  % so it should come on the page before the last. Make
                                  % sure that you do not shorten the textheight too much.

\section*{APPENDIX}

Appendixes should appear before the acknowledgment.

\section*{ACKNOWLEDGMENT}

\begin{thebibliography}{99}

\bibitem{c1} https://www.livescience.com/49007-history-of-artificial-intelligence.html -- A brief history of Artificial Intelligence
\bibitem{c2} https://www.kiseido.com/ff.htm -- Go game 
\bibitem{c3} https://www.hackerearth.com/blog/artificial-intelligence/artificial-intelligence-101-how-to-get-started/ -- Types of A.I






\end{thebibliography}




\end{document}
